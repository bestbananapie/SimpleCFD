\newcommand{\NWtarget}[2]{#2}
\newcommand{\NWlink}[2]{#2}
\newcommand{\NWtxtMacroDefBy}{Fragment defined by}
\newcommand{\NWtxtMacroRefIn}{Fragment referenced in}
\newcommand{\NWtxtMacroNoRef}{Fragment never referenced}
\newcommand{\NWtxtDefBy}{Defined by}
\newcommand{\NWtxtRefIn}{Referenced in}
\newcommand{\NWtxtNoRef}{Not referenced}
\newcommand{\NWtxtFileDefBy}{File defined by}
\newcommand{\NWtxtIdentsUsed}{Uses:}
\newcommand{\NWtxtIdentsNotUsed}{Never used}
\newcommand{\NWtxtIdentsDefed}{Defines:}
\newcommand{\NWsep}{${\diamond}$}
\newcommand{\NWnotglobal}{(not defined globally)}
\newcommand{\NWuseHyperlinks}{}
\documentclass[a4paper,11pt]{article}
\newcommand{\programname}{SimpleCFD }

%Better margins
    \usepackage[left=1in, right=1in, top=1.5in, bottom=1.5in]{geometry}

%Better Font
    \usepackage{lmodern}
    \usepackage[T1]{fontenc}

% Smartter text spacing
    \usepackage{microtype}

% Format paragraphs without indentation
\usepackage[parfill]{parskip}

% Create Todos
\usepackage[colorinlistoftodos,prependcaption,textsize=tiny]{todonotes}

%Create links in PDF
    \usepackage[hidelinks]{hyperref}

\title{\programname}
\author{Simon Lee}
\date{\today}

\begin{document}

\maketitle
\newpage

\tableofcontents
\newpage

\section{Todos}
\listoftodos
\section{Introduction}

I have long had the idea to write my own, but very simple, CFD solver.

My initial goal is the solve for the case of a flow through a rectangular box/duct. The only reason for this is that the grid/mesh should be very easy.


\begin{flushleft} \small
\begin{minipage}{\linewidth}\label{scrap1}\raggedright\small
\NWtarget{nuweb3}{} \verb@"main.c"@\nobreak\ {\footnotesize {3}}$\equiv$
\vspace{-1ex}
\begin{list}{}{} \item
\mbox{}\verb@@\\
\mbox{}\verb@#include "mesh.h"@\\
\mbox{}\verb@#include <stdio.h>@\\
\mbox{}\verb@int main(void){@\\
\mbox{}\verb@    struct_mesh mesh;@\\
\mbox{}\verb@    generate_mesh(&mesh);@\\
\mbox{}\verb@    mesh_print_vtk(&mesh,"test.vtk");@\\
\mbox{}\verb@    printf("Hello World %d %d\n",mesh.num_points, mesh.num_cells);@\\
\mbox{}\verb@    return 0;@\\
\mbox{}\verb@}@\\
\mbox{}\verb@@{\NWsep}
\end{list}
\vspace{-1.5ex}
\footnotesize
\begin{list}{}{\setlength{\itemsep}{-\parsep}\setlength{\itemindent}{-\leftmargin}}

\item{}
\end{list}
\end{minipage}\vspace{4ex}
\end{flushleft}
\newpage
\section{mesh}

\subsection{Introduction}
The code only works with rectagular cells in an unstructured grid. The mesh is defined by a point cloud and each cell has 8 nodes.

Each node is defined by a 3 dimensional vector.


\subsubsection{Data Structure}
\begin{flushleft} \small
\begin{minipage}{\linewidth}\label{scrap2}\raggedright\small
\NWtarget{nuweb4}{} $\langle\,${\itshape mesh-data-structures}\nobreak\ {\footnotesize {4}}$\,\rangle\equiv$
\vspace{-1ex}
\begin{list}{}{} \item
\mbox{}\verb@@\\
\mbox{}\verb@typedef struct {@\\
\mbox{}\verb@    int vtk_type;@\\
\mbox{}\verb@    int *nodes;@\\
\mbox{}\verb@} struct_cell;@\\
\mbox{}\verb@@\\
\mbox{}\verb@typedef struct {@\\
\mbox{}\verb@    int num_points;@\\
\mbox{}\verb@    int num_cells;@\\
\mbox{}\verb@    float scale;@\\
\mbox{}\verb@    int **points;@\\
\mbox{}\verb@    struct_cell *cells;@\\
\mbox{}\verb@    int mem_allocated;@\\
\mbox{}\verb@} struct_mesh;@\\
\mbox{}\verb@@\\
\mbox{}\verb@typedef struct {@\\
\mbox{}\verb@    int num_points;@\\
\mbox{}\verb@    int num_cells;@\\
\mbox{}\verb@    float scale;@\\
\mbox{}\verb@    int **points;@\\
\mbox{}\verb@    struct_cell *cells;@\\
\mbox{}\verb@    int mem_allocated;@\\
\mbox{}\verb@} struct_domain;@\\
\mbox{}\verb@@{\NWsep}
\end{list}
\vspace{-1.5ex}
\footnotesize
\begin{list}{}{\setlength{\itemsep}{-\parsep}\setlength{\itemindent}{-\leftmargin}}
\item \NWtxtMacroRefIn\ \NWlink{nuweb12a}{12a}.

\item{}
\end{list}
\end{minipage}\vspace{4ex}
\end{flushleft}
\subsubsection{Mesh Definition}
\todo{Mesh definition to be input at runtime.}
\begin{flushleft} \small
\begin{minipage}{\linewidth}\label{scrap3}\raggedright\small
\NWtarget{nuweb5}{} $\langle\,${\itshape mesh-def}\nobreak\ {\footnotesize {5}}$\,\rangle\equiv$
\vspace{-1ex}
\begin{list}{}{} \item
\mbox{}\verb@@\\
\mbox{}\verb@    int domain_min[3];@\\
\mbox{}\verb@    int domain_max[3];@\\
\mbox{}\verb@@\\
\mbox{}\verb@    int domain_seedsize[3];@\\
\mbox{}\verb@@\\
\mbox{}\verb@    domain_min[0] = 0;@\\
\mbox{}\verb@    domain_min[1] = 0;@\\
\mbox{}\verb@    domain_min[2] = 0;@\\
\mbox{}\verb@@\\
\mbox{}\verb@    domain_max[0] = 20;@\\
\mbox{}\verb@    domain_max[1] = 10;@\\
\mbox{}\verb@    domain_max[2] = 10;@\\
\mbox{}\verb@@\\
\mbox{}\verb@    /* this is actually half delta */@\\
\mbox{}\verb@    domain_seedsize[0] = 5;@\\
\mbox{}\verb@    domain_seedsize[1] = 5;@\\
\mbox{}\verb@    domain_seedsize[2] = 5;@\\
\mbox{}\verb@@\\
\mbox{}\verb@    mesh->num_points = 0;@\\
\mbox{}\verb@    mesh->num_cells = 0;@\\
\mbox{}\verb@    mesh->scale = 1e-3;@\\
\mbox{}\verb@@{\NWsep}
\end{list}
\vspace{-1.5ex}
\footnotesize
\begin{list}{}{\setlength{\itemsep}{-\parsep}\setlength{\itemindent}{-\leftmargin}}
\item \NWtxtMacroRefIn\ \NWlink{nuweb8}{8}.

\item{}
\end{list}
\end{minipage}\vspace{4ex}
\end{flushleft}
\subsection{Generating Mesh}

\subsubsection{Generating Vertex List}
\begin{flushleft} \small
\begin{minipage}{\linewidth}\label{scrap4}\raggedright\small
\NWtarget{nuweb6}{} $\langle\,${\itshape generate-mesh-vertex}\nobreak\ {\footnotesize {6}}$\,\rangle\equiv$
\vspace{-1ex}
\begin{list}{}{} \item
\mbox{}\verb@@\\
\mbox{}\verb@/* Generate Vertices */@\\
\mbox{}\verb@    p = 0;@\\
\mbox{}\verb@@\\
\mbox{}\verb@    pos[0] = 0;@\\
\mbox{}\verb@    pos[1] = 0;@\\
\mbox{}\verb@    pos[2] = 0;@\\
\mbox{}\verb@@\\
\mbox{}\verb@    while(pos[2] <= domain_max[2]){@\\
\mbox{}\verb@        while(pos[1] <= domain_max[1]){@\\
\mbox{}\verb@            while(pos[0] <= domain_max[0]){@\\
\mbox{}\verb@                points[p][0] = pos[0];@\\
\mbox{}\verb@                points[p][1] = pos[1];@\\
\mbox{}\verb@                points[p][2] = pos[2];@\\
\mbox{}\verb@                p++;@\\
\mbox{}\verb@@\\
\mbox{}\verb@                /* Increment position */@\\
\mbox{}\verb@                pos[0] += domain_seedsize[0];@\\
\mbox{}\verb@                ++vertex_count[0];@\\
\mbox{}\verb@            }@\\
\mbox{}\verb@            pos[1] += domain_seedsize[1];@\\
\mbox{}\verb@            ++vertex_count[1];@\\
\mbox{}\verb@        }@\\
\mbox{}\verb@        pos[2] += domain_seedsize[2];@\\
\mbox{}\verb@        ++vertex_count[2];@\\
\mbox{}\verb@    }@\\
\mbox{}\verb@@{\NWsep}
\end{list}
\vspace{-1.5ex}
\footnotesize
\begin{list}{}{\setlength{\itemsep}{-\parsep}\setlength{\itemindent}{-\leftmargin}}
\item \NWtxtMacroRefIn\ \NWlink{nuweb8}{8}.

\item{}
\end{list}
\end{minipage}\vspace{4ex}
\end{flushleft}
\begin{flushleft} \small
\begin{minipage}{\linewidth}\label{scrap5}\raggedright\small
\NWtarget{nuweb7}{} $\langle\,${\itshape generate-mesh-cells}\nobreak\ {\footnotesize {7}}$\,\rangle\equiv$
\vspace{-1ex}
\begin{list}{}{} \item
\mbox{}\verb@@\\
\mbox{}\verb@/* Create Cells */@\\
\mbox{}\verb@    for(c=0; c < mesh->num_cells ; c++){@\\
\mbox{}\verb@        cells[c].nodes = malloc(9*sizeof(int));@\\
\mbox{}\verb@        cells[c].vtk_type = 11;@\\
\mbox{}\verb@@\\
\mbox{}\verb@        n = 0;@\\
\mbox{}\verb@        while(pos[2] <= domain_max[2]){@\\
\mbox{}\verb@            while(pos[1] <= domain_max[1]){@\\
\mbox{}\verb@                while(pos[0] <= domain_max[0]){@\\
\mbox{}\verb@                    xindex = (int) (x-xmin)/(2*dx);  @\\
\mbox{}\verb@                    yindex = (int) (y-ymin)/(2*dy);  @\\
\mbox{}\verb@                    zindex = (int) (z-zmin)/(2*dz);  @\\
\mbox{}\verb@                    cells[c].nodes[n]  = zindex * (xcount+1) * (ycount*1);@\\
\mbox{}\verb@                    cells[c].nodes[n] += yindex * (xcount+1);@\\
\mbox{}\verb@                    cells[c].nodes[n] += xindex;@\\
\mbox{}\verb@                    n++;@\\
\mbox{}\verb@                }@\\
\mbox{}\verb@            }@\\
\mbox{}\verb@        }@\\
\mbox{}\verb@    }@\\
\mbox{}\verb@@{\NWsep}
\end{list}
\vspace{-1.5ex}
\footnotesize
\begin{list}{}{\setlength{\itemsep}{-\parsep}\setlength{\itemindent}{-\leftmargin}}
\item \NWtxtMacroRefIn\ \NWlink{nuweb8}{8}.

\item{}
\end{list}
\end{minipage}\vspace{4ex}
\end{flushleft}
\begin{flushleft} \small
\begin{minipage}{\linewidth}\label{scrap6}\raggedright\small
\NWtarget{nuweb8}{} $\langle\,${\itshape func-generate-mesh}\nobreak\ {\footnotesize {8}}$\,\rangle\equiv$
\vspace{-1ex}
\begin{list}{}{} \item
\mbox{}\verb@@\\
\mbox{}\verb@@\\
\mbox{}\verb@int generate_mesh(struct_domain * domain, struct_mesh * mesh){@\\
\mbox{}\verb@    int n,c,p;@\\
\mbox{}\verb@@\\
\mbox{}\verb@    int pos[3];@\\
\mbox{}\verb@@\\
\mbox{}\verb@    int xindex;@\\
\mbox{}\verb@    int yindex;@\\
\mbox{}\verb@    int zindex;@\\
\mbox{}\verb@@\\
\mbox{}\verb@    int **points;@\\
\mbox{}\verb@    struct_cell *cells;@\\
\mbox{}\verb@@\\
\mbox{}\verb@    int vertex_count[0] = 0;@\\
\mbox{}\verb@    int vertex_count[1] = 0;@\\
\mbox{}\verb@    int vertex_count[2] = 0;@\\
\mbox{}\verb@@\\
\mbox{}\verb@    @\hbox{$\langle\,${\itshape mesh-def}\nobreak\ {\footnotesize \NWlink{nuweb5}{5}}$\,\rangle$}\verb@@\\
\mbox{}\verb@@\\
\mbox{}\verb@    printf("mesh points %d\n", mesh->num_points);@\\
\mbox{}\verb@    printf("mesh cells %d\n", mesh->num_cells);@\\
\mbox{}\verb@    @\\
\mbox{}\verb@/* Mememory Allocation */@\\
\mbox{}\verb@    @\\
\mbox{}\verb@    // Lets guess a mesh size...@\\
\mbox{}\verb@    mesh->num_points = (vertex_count[0]+1)*(vertex_count[1]+1)*(vertex_count[2]+1)@\\
\mbox{}\verb@    mesh->num_cells = xcount * ycount * zcount;@\\
\mbox{}\verb@@\\
\mbox{}\verb@    //... and allocate it@\\
\mbox{}\verb@    mesh->mem_allocated = 0;@\\
\mbox{}\verb@    mesh_mem_allocate(mesh);@\\
\mbox{}\verb@    @\\
\mbox{}\verb@@\\
\mbox{}\verb@    @\hbox{$\langle\,${\itshape generate-mesh-vertex}\nobreak\ {\footnotesize \NWlink{nuweb6}{6}}$\,\rangle$}\verb@@\\
\mbox{}\verb@@\\
\mbox{}\verb@    @\hbox{$\langle\,${\itshape generate-mesh-cells}\nobreak\ {\footnotesize \NWlink{nuweb7}{7}}$\,\rangle$}\verb@@\\
\mbox{}\verb@@\\
\mbox{}\verb@    mesh->points = points;@\\
\mbox{}\verb@    mesh->cells = cells;@\\
\mbox{}\verb@    return 0;@\\
\mbox{}\verb@}@\\
\mbox{}\verb@@\\
\mbox{}\verb@@{\NWsep}
\end{list}
\vspace{-1.5ex}
\footnotesize
\begin{list}{}{\setlength{\itemsep}{-\parsep}\setlength{\itemindent}{-\leftmargin}}
\item \NWtxtMacroRefIn\ \NWlink{nuweb12b}{12b}.

\item{}
\end{list}
\end{minipage}\vspace{4ex}
\end{flushleft}
\subsubsection{Dynamic Memory Allocation}

\paragraph{Memory Allocation}
Memory for the mesh get allocations based on the number of vertices and cells stored within the mesh data structure.
\todo{Reallocate memory for mesh}

\begin{flushleft} \small
\begin{minipage}{\linewidth}\label{scrap7}\raggedright\small
\NWtarget{nuweb9a}{} $\langle\,${\itshape mesh-mem-allocation}\nobreak\ {\footnotesize {9a}}$\,\rangle\equiv$
\vspace{-1ex}
\begin{list}{}{} \item
\mbox{}\verb@/* Allocate memory for mesh */@\\
\mbox{}\verb@int mesh_mem_allocate(struct_mesh * mesh){@\\
\mbox{}\verb@@\\
\mbox{}\verb@    if(mesh->mem_allocated ==0){@\\
\mbox{}\verb@        /* Allocate Memeory for Points */@\\
\mbox{}\verb@        mesh->points = malloc(mesh->num_points*sizeof(*points) + mesh->num_points*3*sizeof(**points));@\\
\mbox{}\verb@@\\
\mbox{}\verb@        int *data;@\\
\mbox{}\verb@        data = &points[mesh->num_points];@\\
\mbox{}\verb@        for(n=0; n < mesh->num_points; n++){@\\
\mbox{}\verb@            points[n] = data + n * 3;@\\
\mbox{}\verb@        }@\\
\mbox{}\verb@@\\
\mbox{}\verb@        /* Allocate Memeory for Cells */@\\
\mbox{}\verb@        cells = malloc(mesh->num_cells*sizeof(*cells));@\\
\mbox{}\verb@@\\
\mbox{}\verb@        mesh->mem_allocated = 1;@\\
\mbox{}\verb@    }@\\
\mbox{}\verb@    else {@\\
\mbox{}\verb@        /* Reallocate Memory */@\\
\mbox{}\verb@    }@\\
\mbox{}\verb@   @\\
\mbox{}\verb@   return 0;@\\
\mbox{}\verb@}@\\
\mbox{}\verb@@{\NWsep}
\end{list}
\vspace{-1.5ex}
\footnotesize
\begin{list}{}{\setlength{\itemsep}{-\parsep}\setlength{\itemindent}{-\leftmargin}}
\item \NWtxtMacroRefIn\ \NWlink{nuweb12b}{12b}.

\item{}
\end{list}
\end{minipage}\vspace{4ex}
\end{flushleft}
\paragraph{Free Memory}
\begin{flushleft} \small
\begin{minipage}{\linewidth}\label{scrap8}\raggedright\small
\NWtarget{nuweb9b}{} $\langle\,${\itshape mesh-mem-free}\nobreak\ {\footnotesize {9b}}$\,\rangle\equiv$
\vspace{-1ex}
\begin{list}{}{} \item
\mbox{}\verb@/* Free memory allocated to mesh */@\\
\mbox{}\verb@int mesh_mem_free(struct_mesh * mesh){@\\
\mbox{}\verb@@\\
\mbox{}\verb@   free(mesh->points)@\\
\mbox{}\verb@   free(mesh->cells)@\\
\mbox{}\verb@   return 0;@\\
\mbox{}\verb@}@\\
\mbox{}\verb@@{\NWsep}
\end{list}
\vspace{-1.5ex}
\footnotesize
\begin{list}{}{\setlength{\itemsep}{-\parsep}\setlength{\itemindent}{-\leftmargin}}
\item \NWtxtMacroRefIn\ \NWlink{nuweb12b}{12b}.

\item{}
\end{list}
\end{minipage}\vspace{4ex}
\end{flushleft}
\subsection{Exporting Mesh}
The mesh can be exported to a vtk file format to be read in Paraview.

\begin{flushleft} \small
\begin{minipage}{\linewidth}\label{scrap9}\raggedright\small
\NWtarget{nuweb10a}{} $\langle\,${\itshape func-mesh-print-vtk}\nobreak\ {\footnotesize {10a}}$\,\rangle\equiv$
\vspace{-1ex}
\begin{list}{}{} \item
\mbox{}\verb@#Export mesh to VTK@\\
\mbox{}\verb@int mesh_print_vtk(struct_mesh *mesh, char *filename) {@\\
\mbox{}\verb@    int n;@\\
\mbox{}\verb@    FILE * fp;@\\
\mbox{}\verb@    fp = fopen(filename,"w");@\\
\mbox{}\verb@    @\\
\mbox{}\verb@@\hbox{$\langle\,${\itshape snippet-mesh-print-vtk-header}\nobreak\ {\footnotesize \NWlink{nuweb10b}{10b}}$\,\rangle$}\verb@@\\
\mbox{}\verb@@\\
\mbox{}\verb@@\hbox{$\langle\,${\itshape snippt-mesh-print-vtk-vertices}\nobreak\ {\footnotesize \NWlink{nuweb11a}{11a}}$\,\rangle$}\verb@@\\
\mbox{}\verb@@\\
\mbox{}\verb@@\hbox{$\langle\,${\itshape snippt-mesh-print-vtk-cells}\nobreak\ {\footnotesize \NWlink{nuweb11b}{11b}}$\,\rangle$}\verb@@\\
\mbox{}\verb@@\\
\mbox{}\verb@    fclose(fp);@\\
\mbox{}\verb@@\\
\mbox{}\verb@    return 0;@\\
\mbox{}\verb@    }@\\
\mbox{}\verb@@{\NWsep}
\end{list}
\vspace{-1.5ex}
\footnotesize
\begin{list}{}{\setlength{\itemsep}{-\parsep}\setlength{\itemindent}{-\leftmargin}}
\item \NWtxtMacroRefIn\ \NWlink{nuweb12b}{12b}.

\item{}
\end{list}
\end{minipage}\vspace{4ex}
\end{flushleft}
\subsubsection{VTK Header}
It currently defaults to an unstructed grid
\begin{flushleft} \small
\begin{minipage}{\linewidth}\label{scrap10}\raggedright\small
\NWtarget{nuweb10b}{} $\langle\,${\itshape snippet-mesh-print-vtk-header}\nobreak\ {\footnotesize {10b}}$\,\rangle\equiv$
\vspace{-1ex}
\begin{list}{}{} \item
\mbox{}\verb@/* Print file header */@\\
\mbox{}\verb@    fprintf(fp, "# vtk DataFile Version 2.0\n"@\\
\mbox{}\verb@                "SimpleCFD VTK Output\n"@\\
\mbox{}\verb@                "ASCII\n"@\\
\mbox{}\verb@                "DATASET UNSTRUCTURED_GRID\n");@\\
\mbox{}\verb@@{\NWsep}
\end{list}
\vspace{-1.5ex}
\footnotesize
\begin{list}{}{\setlength{\itemsep}{-\parsep}\setlength{\itemindent}{-\leftmargin}}
\item \NWtxtMacroRefIn\ \NWlink{nuweb10a}{10a}.

\item{}
\end{list}
\end{minipage}\vspace{4ex}
\end{flushleft}
\subsubsection{Print Vertices}
\begin{flushleft} \small
\begin{minipage}{\linewidth}\label{scrap11}\raggedright\small
\NWtarget{nuweb11a}{} $\langle\,${\itshape snippt-mesh-print-vtk-vertices}\nobreak\ {\footnotesize {11a}}$\,\rangle\equiv$
\vspace{-1ex}
\begin{list}{}{} \item
\mbox{}\verb@/* Print Vertices */@\\
\mbox{}\verb@    fprintf(fp, "\nPOINTS %d FLOAT\n", mesh->num_points);@\\
\mbox{}\verb@@\\
\mbox{}\verb@    for(n=0; n < mesh->num_points; n++){@\\
\mbox{}\verb@        fprintf(fp, "%+e %+e %+e\n", (float) mesh->points[n][0] * mesh->scale,@\\
\mbox{}\verb@                                     (float) mesh->points[n][1] * mesh->scale,@\\
\mbox{}\verb@                                     (float) mesh->points[n][2] * mesh->scale);@\\
\mbox{}\verb@    }@\\
\mbox{}\verb@@{\NWsep}
\end{list}
\vspace{-1.5ex}
\footnotesize
\begin{list}{}{\setlength{\itemsep}{-\parsep}\setlength{\itemindent}{-\leftmargin}}
\item \NWtxtMacroRefIn\ \NWlink{nuweb10a}{10a}.

\item{}
\end{list}
\end{minipage}\vspace{4ex}
\end{flushleft}
\subsubsection{Printing Cells Information}
Each cell is defined by 8 vertices. Type 12 refers to a VTK relationship between nodes.

\begin{flushleft} \small
\begin{minipage}{\linewidth}\label{scrap12}\raggedright\small
\NWtarget{nuweb11b}{} $\langle\,${\itshape snippt-mesh-print-vtk-cells}\nobreak\ {\footnotesize {11b}}$\,\rangle\equiv$
\vspace{-1ex}
\begin{list}{}{} \item
\mbox{}\verb@/* Print Cell Information */@\\
\mbox{}\verb@    fprintf(fp, "\nCELLS %d %d \n", mesh->num_cells, mesh->num_cells * 9);@\\
\mbox{}\verb@@\\
\mbox{}\verb@    for(n=0; n < mesh->num_cells; n++){@\\
\mbox{}\verb@        fprintf(fp, "8 %d %d %d %d %d %d %d %d\n", mesh->cells[n].nodes[0],@\\
\mbox{}\verb@                                                   mesh->cells[n].nodes[1],@\\
\mbox{}\verb@                                                   mesh->cells[n].nodes[2],@\\
\mbox{}\verb@                                                   mesh->cells[n].nodes[3],@\\
\mbox{}\verb@                                                   mesh->cells[n].nodes[4],@\\
\mbox{}\verb@                                                   mesh->cells[n].nodes[5],@\\
\mbox{}\verb@                                                   mesh->cells[n].nodes[6],@\\
\mbox{}\verb@                                                   mesh->cells[n].nodes[7]);@\\
\mbox{}\verb@    }@\\
\mbox{}\verb@@\\
\mbox{}\verb@/* Print Cell Types */@\\
\mbox{}\verb@    fprintf(fp, "\nCELL_TYPES %d \n", mesh->num_cells);@\\
\mbox{}\verb@@\\
\mbox{}\verb@    for(n=0; n < mesh->num_cells; n++){@\\
\mbox{}\verb@        fprintf(fp, "12 \n");@\\
\mbox{}\verb@    }@\\
\mbox{}\verb@@{\NWsep}
\end{list}
\vspace{-1.5ex}
\footnotesize
\begin{list}{}{\setlength{\itemsep}{-\parsep}\setlength{\itemindent}{-\leftmargin}}
\item \NWtxtMacroRefIn\ \NWlink{nuweb10a}{10a}.

\item{}
\end{list}
\end{minipage}\vspace{4ex}
\end{flushleft}
\subsection{Source file}
All the functions get written to a common source file.

\begin{flushleft} \small
\begin{minipage}{\linewidth}\label{scrap13}\raggedright\small
\NWtarget{nuweb12a}{} \verb@"mesh.h"@\nobreak\ {\footnotesize {12a}}$\equiv$
\vspace{-1ex}
\begin{list}{}{} \item
\mbox{}\verb@@\\
\mbox{}\verb@#ifndef MESH_H@\\
\mbox{}\verb@#define MESH_H@\\
\mbox{}\verb@@\\
\mbox{}\verb@@\hbox{$\langle\,${\itshape mesh-data-structures}\nobreak\ {\footnotesize \NWlink{nuweb4}{4}}$\,\rangle$}\verb@@\\
\mbox{}\verb@@\\
\mbox{}\verb@int generate_mesh(struct_mesh * mesh);@\\
\mbox{}\verb@@\\
\mbox{}\verb@int mesh_print_vtk(struct_mesh *mesh, char *filename);@\\
\mbox{}\verb@@\\
\mbox{}\verb@#endif@\\
\mbox{}\verb@@{\NWsep}
\end{list}
\vspace{-1.5ex}
\footnotesize
\begin{list}{}{\setlength{\itemsep}{-\parsep}\setlength{\itemindent}{-\leftmargin}}

\item{}
\end{list}
\end{minipage}\vspace{4ex}
\end{flushleft}
\begin{flushleft} \small
\begin{minipage}{\linewidth}\label{scrap14}\raggedright\small
\NWtarget{nuweb12b}{} \verb@"mesh.c"@\nobreak\ {\footnotesize {12b}}$\equiv$
\vspace{-1ex}
\begin{list}{}{} \item
\mbox{}\verb@@\\
\mbox{}\verb@#include "mesh.h" @\\
\mbox{}\verb@#include <stdio.h>@\\
\mbox{}\verb@#include <stdlib.h>@\\
\mbox{}\verb@#include <math.h>@\\
\mbox{}\verb@@\\
\mbox{}\verb@@\hbox{$\langle\,${\itshape mesh-mem-free}\nobreak\ {\footnotesize \NWlink{nuweb9b}{9b}}$\,\rangle$}\verb@@\\
\mbox{}\verb@@\\
\mbox{}\verb@@\hbox{$\langle\,${\itshape mesh-mem-allocation}\nobreak\ {\footnotesize \NWlink{nuweb9a}{9a}}$\,\rangle$}\verb@@\\
\mbox{}\verb@@\\
\mbox{}\verb@@\hbox{$\langle\,${\itshape func-generate-mesh}\nobreak\ {\footnotesize \NWlink{nuweb8}{8}}$\,\rangle$}\verb@@\\
\mbox{}\verb@@\\
\mbox{}\verb@@\hbox{$\langle\,${\itshape func-mesh-print-vtk}\nobreak\ {\footnotesize \NWlink{nuweb10a}{10a}}$\,\rangle$}\verb@@\\
\mbox{}\verb@@{\NWsep}
\end{list}
\vspace{-1.5ex}
\footnotesize
\begin{list}{}{\setlength{\itemsep}{-\parsep}\setlength{\itemindent}{-\leftmargin}}

\item{}
\end{list}
\end{minipage}\vspace{4ex}
\end{flushleft}
\section{preprocessor}
\begin{flushleft} \small
\begin{minipage}{\linewidth}\label{scrap15}\raggedright\small
\NWtarget{nuweb12c}{} $\langle\,${\itshape preprocessor.c}\nobreak\ {\footnotesize {12c}}$\,\rangle\equiv$
\vspace{-1ex}
\begin{list}{}{} \item
\mbox{}\verb@@\\
\mbox{}\verb@@{\NWsep}
\end{list}
\vspace{-1.5ex}
\footnotesize
\begin{list}{}{\setlength{\itemsep}{-\parsep}\setlength{\itemindent}{-\leftmargin}}
\item {\NWtxtMacroNoRef}.

\item{}
\end{list}
\end{minipage}\vspace{4ex}
\end{flushleft}
\section{postprocessor}
\begin{flushleft} \small
\begin{minipage}{\linewidth}\label{scrap16}\raggedright\small
\NWtarget{nuweb13a}{} $\langle\,${\itshape postprocessor.c}\nobreak\ {\footnotesize {13a}}$\,\rangle\equiv$
\vspace{-1ex}
\begin{list}{}{} \item
\mbox{}\verb@@\\
\mbox{}\verb@@{\NWsep}
\end{list}
\vspace{-1.5ex}
\footnotesize
\begin{list}{}{\setlength{\itemsep}{-\parsep}\setlength{\itemindent}{-\leftmargin}}
\item {\NWtxtMacroNoRef}.

\item{}
\end{list}
\end{minipage}\vspace{4ex}
\end{flushleft}
\section{Solver}

\begin{flushleft} \small
\begin{minipage}{\linewidth}\label{scrap17}\raggedright\small
\NWtarget{nuweb13b}{} $\langle\,${\itshape solver.c}\nobreak\ {\footnotesize {13b}}$\,\rangle\equiv$
\vspace{-1ex}
\begin{list}{}{} \item
\mbox{}\verb@@\\
\mbox{}\verb@@\\
\mbox{}\verb@@\\
\mbox{}\verb@@\\
\mbox{}\verb@@{\NWsep}
\end{list}
\vspace{-1.5ex}
\footnotesize
\begin{list}{}{\setlength{\itemsep}{-\parsep}\setlength{\itemindent}{-\leftmargin}}
\item {\NWtxtMacroNoRef}.

\item{}
\end{list}
\end{minipage}\vspace{4ex}
\end{flushleft}
\appendix

\section{Utilities}
To automate working with \programname, a few utilities have been created.

\programname uses GNU Make to generate all the required files. It is used to
produce to produce the document and source files and compile as required. 

\subsection{Root Makefile}
The makefile that sits in the root directory of the project has targets to allow
it to compile the source, documentation and clean temporary files. By default it
outputs the source code.


%supress tab expansion and scrap indentation
\begin{flushleft} \small
\begin{minipage}{\linewidth}\label{scrap18}\raggedright\small
\NWtarget{nuweb13c}{} \verb@"../Makefile"@\nobreak\ {\footnotesize {13c}}$\equiv$
\vspace{-1ex}
\begin{list}{}{} \item
\mbox{}\verb@@\\
\mbox{}\verb@#Define phony targets@\\
\mbox{}\verb@.PHONY: all doc@\\
\mbox{}\verb@@\\
\mbox{}\verb@# Output the Source Code@\\
\mbox{}\verb@all: SimpleCFD.w@\\
\mbox{}\verb@        @\hbox{$\langle\,${\itshape make-src}\nobreak\ {\footnotesize \NWlink{nuweb14a}{14a}}$\,\rangle$}\verb@@\\
\mbox{}\verb@@\\
\mbox{}\verb@# Make the documents@\\
\mbox{}\verb@doc: SimpleCFD.w@\\
\mbox{}\verb@        @\hbox{$\langle\,${\itshape make-doc}\nobreak\ {\footnotesize \NWlink{nuweb14b}{14b}}$\,\rangle$}\verb@@\\
\mbox{}\verb@@{\NWsep}
\end{list}
\vspace{-1.5ex}
\footnotesize
\begin{list}{}{\setlength{\itemsep}{-\parsep}\setlength{\itemindent}{-\leftmargin}}

\item{}
\end{list}
\end{minipage}\vspace{4ex}
\end{flushleft}
\subsubsection{Source Output}
The all target 'TANGLES' the source text into the src folder with nuweb and
executes make within the source folder. It also explicitly scaffolds folders as necessary. 
\begin{flushleft} \small
\begin{minipage}{\linewidth}\label{scrap19}\raggedright\small
\NWtarget{nuweb14a}{} $\langle\,${\itshape make-src}\nobreak\ {\footnotesize {14a}}$\,\rangle\equiv$
\vspace{-1ex}
\begin{list}{}{} \item
\mbox{}\verb@@\\
\mbox{}\verb@        mkdir -p src@\\
\mbox{}\verb@        nuweb -t -p src $^@\\
\mbox{}\verb@        $(MAKE) -C src@\\
\mbox{}\verb@@{\NWsep}
\end{list}
\vspace{-1.5ex}
\footnotesize
\begin{list}{}{\setlength{\itemsep}{-\parsep}\setlength{\itemindent}{-\leftmargin}}
\item \NWtxtMacroRefIn\ \NWlink{nuweb13c}{13c}.

\item{}
\end{list}
\end{minipage}\vspace{4ex}
\end{flushleft}
\subsubsection{Documentation Output}
The documentation is produced by `weaving' the source text. All the
documentation output is generated into a separation folder.

\begin{flushleft} \small
\begin{minipage}{\linewidth}\label{scrap20}\raggedright\small
\NWtarget{nuweb14b}{} $\langle\,${\itshape make-doc}\nobreak\ {\footnotesize {14b}}$\,\rangle\equiv$
\vspace{-1ex}
\begin{list}{}{} \item
\mbox{}\verb@@\\
\mbox{}\verb@        mkdir -p doc@\\
\mbox{}\verb@        nuweb -o -p doc $^@\\
\mbox{}\verb@        pdflatex --output-directory doc $(basename $^).tex@\\
\mbox{}\verb@@{\NWsep}
\end{list}
\vspace{-1.5ex}
\footnotesize
\begin{list}{}{\setlength{\itemsep}{-\parsep}\setlength{\itemindent}{-\leftmargin}}
\item \NWtxtMacroRefIn\ \NWlink{nuweb13c}{13c}.

\item{}
\end{list}
\end{minipage}\vspace{4ex}
\end{flushleft}
Note: For all the cross references to be picked up correctly, this needs to be run at
least twice.

\section{Building and Compilation of Source}
The source code is compiled using GNU Make.

\subsection{Compile Configurations}
The configuration options of the complilation step sits in common folder to be
sharded across make files.

\begin{flushleft} \small
\begin{minipage}{\linewidth}\label{scrap21}\raggedright\small
\NWtarget{nuweb15a}{} \verb@"config.mk"@\nobreak\ {\footnotesize {15a}}$\equiv$
\vspace{-1ex}
\begin{list}{}{} \item
\mbox{}\verb@# Common Configuration File@\\
\mbox{}\verb@@\\
\mbox{}\verb@## Default Compilers@\\
\mbox{}\verb@CXX=g++@\\
\mbox{}\verb@CC=gcc@\\
\mbox{}\verb@@\\
\mbox{}\verb@@\\
\mbox{}\verb@## Flags for the C compiler@\\
\mbox{}\verb@CCFLAGS  = -g @\\
\mbox{}\verb@CCFLAGS += -Wpointer-arith@\\
\mbox{}\verb@CCFLAGS += -Wshadow -Winit-self@\\
\mbox{}\verb@CCFLAGS += -Wextra@\\
\mbox{}\verb@CCFLAGS += -Wfloat-equal@\\
\mbox{}\verb@CCFLAGS += -Wall@\\
\mbox{}\verb@CCFLAGS += -std=c99 @\\
\mbox{}\verb@CCFLAGS += -pedantic@\\
\mbox{}\verb@CCFLAGS += -O3@\\
\mbox{}\verb@@{\NWsep}
\end{list}
\vspace{-1.5ex}
\footnotesize
\begin{list}{}{\setlength{\itemsep}{-\parsep}\setlength{\itemindent}{-\leftmargin}}

\item{}
\end{list}
\end{minipage}\vspace{4ex}
\end{flushleft}
\subsection{Comiling Source}
\begin{flushleft} \small
\begin{minipage}{\linewidth}\label{scrap22}\raggedright\small
\NWtarget{nuweb15b}{} \verb@"Makefile"@\nobreak\ {\footnotesize {15b}}$\equiv$
\vspace{-1ex}
\begin{list}{}{} \item
\mbox{}\verb@#Makefile for Codebase@\\
\mbox{}\verb@@\\
\mbox{}\verb@## Include the common configuration file@\\
\mbox{}\verb@include config.mk@\\
\mbox{}\verb@@\\
\mbox{}\verb@##Define phony targets@\\
\mbox{}\verb@.PHONY: all clean@\\
\mbox{}\verb@@\\
\mbox{}\verb@SOURCES=$(wildcard *.c)@\\
\mbox{}\verb@OBJECTS=$(SOURCES:.c=.o)@\\
\mbox{}\verb@TARGET=../simplecfd@\\
\mbox{}\verb@@\\
\mbox{}\verb@all: $(TARGET)@\\
\mbox{}\verb@@\\
\mbox{}\verb@$(TARGET): $(OBJECTS)@\\
\mbox{}\verb@        $(CC) -o $@{\tt @}\verb@ $^@\\
\mbox{}\verb@@\\
\mbox{}\verb@%.o: %.c@\\
\mbox{}\verb@        $(CC) $(CCFLAGS) -c $<@\\
\mbox{}\verb@@{\NWsep}
\end{list}
\vspace{-1.5ex}
\footnotesize
\begin{list}{}{\setlength{\itemsep}{-\parsep}\setlength{\itemindent}{-\leftmargin}}

\item{}
\end{list}
\end{minipage}\vspace{4ex}
\end{flushleft}
\end{document}
